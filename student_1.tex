\section{Individuelle Zielsetzung des Teammitglieds „Josef
Mustermann“}\label{sec:student1main}
\subsection{Aufgabenstellung}\label{sec:student1tasks}

Beschreiben Sie die Aufgabenstellung zu Ihrer individuellen Zielsetzung.
\subsection{Grundlagen und Methoden}\label{sec:student1methods}

Beschreiben Sie theoretisch den Stand der Technik, die Untersuchung der möglichen Lösungsalternativen und begründen Sie die Wahl des Lösungsansatzes, die auf Ihre individuelle Zielsetzung zutrifft (korrespondierend zu Punkt 3). 

\subsection{Realisierung}\label{sec:studentAwork}

Beschreiben Sie alle theoretischen Betrachtungen und praktischen Tätigkeiten, die Sie während der Umsetzung Ihrer individuellen Zielsetzung durchgeführt haben. 
Hier erhält die Leserin und der Leser eine Antwort auf die Frage: „Was habe ich, Josef Mustermann, alles gemacht, um meine individuelle Zielsetzung zu erreichen?“

\subsection{Ergebnisse}\label{sec:student1results}

\subsubsection{Beispiel}
Analyse des technischen Istzustandes (kann auch einen theoretischen Anteil
haben) Dimensionierung und Simulation einer elektronischen Schaltung
Entwicklung, Inbetriebnahme, Tests
Darstellung und Diskussion entsprechender Messergebnisse
Diverse Schnittstellen (sowohl Software als auch Hardware)
Gehäuse (CAD – Zeichnungen, Fotos)
Testabläufe (Testfälle bei SW), Testergebnisse, etc.

